\section{Estructura del Repositorio}
El repositorio ha sido organizado de la siguiente manera:
\begin{lstlisting}
DuckWordle/
|-- assets/
    |-- data/
        duck_words.txt
        es_words.txt
    |-- images/
        check.png
        duck.png
        erase.png
    |-- sounds/
        quack.mp3
        win.mp3
|-- core/
    |-- classes.py
    |-- utils.py
|-- docs/
    |-- assets/
        |-- duck.png
    |-- config/
        |-- commands.tex
        |-- usepackages.tex
    |-- sections/
        |-- about.tex
        |-- algorithm.tex
        |-- front.tex
        |-- license.tex
        |-- structure.tex
    |-- Duck\_Wordle.pdf
    |-- main.tex
|-- .gitignore
|-- LICENSE
|-- README-ES.md
|-- README.md
|-- index.py
|-- requirements.txt
\end{lstlisting}

\subsection{Descripción de Carpetas y Archivos}

\begin{itemize}
    \item \textbf{assets/}: Archivos multimedia y de texto.
    \item \textbf{core/}: Código fuente principal del proyecto.
    \item \textbf{docs/}: Documentación del proyecto (incluye este archivo \LaTeX).
    \item \textbf{.gitignore}: Archivo que ignora cache, variables de entorno, etc.
    \item \textbf{README-ES.md}: Archivo de introducción al proyecto en español.
    \item \textbf{README.md}: Archivo de introducción al proyecto en inglés.
    \item \textbf{index.py}: Archivo principal de ejecución del proyecto.
    \item \textbf{requirements.txt}: Dependencias del entorno.
\end{itemize}